\begin{enumerate}[leftmargin=0.8cm]
    \item[a)]{Actividades profesionales fuera del ámbito académico.
        \\No corresponde
    }

    \item[b)]{Actividades de desarrollo tecnológico realizadas fuera del ámbito académico o como parte de proyectos de transferencia entre el sector Científico y el Productivo
        \\No corresponde
    }

    \item[c)]{Otros antecedentes vinculados a la actividad profesional (no académica).

    \begin{itemize}[leftmargin=0.2cm]

      \WorkEntry{\textbf{Software Developer}}
      {Interview Query}
      {Desde Julio 2021 (2 años, en curso)}
      {
        Plataforma para aprender y practicar Data Science: \url{https://www.interviewquery.com}. Mi rol principal es el desarrollo fullstack de la plataforma en sí misma. Algunos proyectos realizados:
        \\ - Servicio que ejecuta y valida las respuestas de los usuarios (soporta Python, R, MySQL, PostgreSQL).
        \\ - Desarrollé un CMS propio para la confección de artículos embebidos con gráficos interactivos alimentados por nuestra propia base de datos y análisis.
        \\ - Migración a Next.js, frontend prácticamente rehecho desde cero.
        \\ - Sistema de ejecución de jobs tipo ETL (Python).
        \\ - Videochat desarrollado con WebRTC.
        \\ Tecnologías: Go, Python, TypeScript, React, MySQL, AWS.
      }

      ~

      \WorkEntry{\textbf{Software Developer}}
      {Mantra Ventures / Acrylic Idea Factory}
      {Desde Diciembre 2021 a Junio 2022 (6 meses)}
      {
        La empresa vende productos de acrílico grabados con imágenes y/o texto. Mis tareas son desarrollar APIs para integrar con otros sistemas y procesar pedidos de forma automatizada.
        \\ Tecnologías: Python, Django, Redis, MySQL.
      }

      ~

      \WorkEntry{\textbf{CTO \& Co-Fundador}}
      {Magoya}
      {Desde Mayo 2017 a Marzo 2021 (4 años)}
      {
        Co-fundé Magoya (\url{https://magoya.com}) para proveer servicios de desarrollo de software. Trabajamos principalmente en la industria agrícola para clientes como Bayer, Nidera, Syngenta, Precision Planting y Red Surcos.
        \\ \\ Responsabilidades:
        \\ - Relevar y documentar las necesidades del cliente y traducirlas a tareas accionables por el equipo.
        \\ - Definir la tecnología y arquitectura de todos los proyectos.
        \\ - Guiar y capacitar al equipo de desarrollo con buenas prácticas.
        \\ - Mucha programación, siempre me mantuve cerca del código.
        \\ - Configurar y mantener la infraestructura en la nube y los pipelines de CI/CD.
        \\ - Ayudar a implementar metodologías ágiles para organizar al equipo (Scrum/Kanban).
        \\ \\ Proyectos: Cultivio (\url{https://cultivio.com}), Sappio Quoting App (\url{https://sappio.app}), Nidera SSO, Nidera Quoting App, Nidera Marketing Portal, Precision Planting Order Guide, Red Surcos Quoting App, Indigo Data Platform, Smart Safety.
        \\ \\ Tecnologías: Python, Django, Django Rest Framework, GraphQL, TypeScript, React, Next.js, Material UI, PostgreSQL, PostGIS, Redis, CouchDB, Nginx, Kubernetes, Google Cloud.
      }

      ~

      \WorkEntry{\textbf{Software Developer}}
      {Factom}
      {Desde Diciembre 2016 a Octubre 2017 (1 año)}
      {
        Desarrollé varias webapps que integran con la plataforma "blockchain as a service" de Factom.
        \\ \\ Responsabilidades:
        \\ - Programación.
        \\ - Reuniones diarias con el product owner (en inglés).
        \\ - Relevar requerimientos y ayudar a definir el producto.
        \\ - Coordinar el equipo local de Argentina (1 desarrollador y 2 diseñadores).
        \\ \\ Proyectos:
        \\ - Factom Explorer: Blockchain explorer app (desarrollé la versión inicial que ya no está online).
        \\ - Factom Harmony UI: App para manejar el proceso de hipotecas (colecta de documentación, validación, etc).
        \\ - Factom Gates UI: App demo para mostrar un posible caso de uso para la blockchain de Factom.
        \\ \\ Tecnologías: JavaScript, React, Redux, Mocha, Enzyme, PouchDB.
      }

      ~

      \WorkEntry{\textbf{Software Developer}}
      {Cielo24}
      {Desde Mayo 2016 a Abril 2017 (1 año)}
      {
        Desarrollé una webapp para visualizar transcripciones de videos sincronizados con la reproducción del mismo. Además permite al usuario buscar dentro de la transcripción y muestra meta-información adicional como ser las personas en el video y los temas mencionados.
        \\ \\ Responsabilidades:
        \\ - Programación.
        \\ - Reuniones diarias con el product owner (en inglés).
        \\ - Relevar requerimientos y ayudar a definir el producto.
        \\ - Coordinar el equipo local de Argentina (1 desarrollador y 2 diseñadores).
        \\ \\ Proyectos:
        \\ - Cielo24 Video Wrapper: \url{https://vwrap.cielo24.com}
        \\ \\ Tecnologías: JavaScript, React, Redux, Immutable.js, Mocha, Enzyme.
      }

      ~

      \WorkEntry{\textbf{Software Developer}}
      {LiderProp}
      {Desde Noviembre 2013 a Abril 2017 (3.5 años)}
      {
        Desarrollé un buscador de inmuebles: Liderprop (\url{https://liderprop.com}).
        Otro equipo ha continuado el proyecto desde el 2017, pero gran parte del código es el mismo que escribí yo en la versión inicial de la plataforma.
        \\ \\ Responsabilidades:
        \\ - Relevar requerimientos y definir el producto.
        \\ - Definir la tecnología y arquitectura de la plataforma.
        \\ - Programación backend y frontend.
        \\ - Configuración de la infraestructura.
        \\ \\ Módulos principales:
        \\ - Buscador de propiedades.
        \\ - Perfil de inmobiliaria y administración de sus propiedades.
        \\ - Perfil de usuario, búsquedas guardadas, alertas automáticas.
        \\ - Integraciones con otras plataformas para sincronizar propiedades entre buscadores.
        \\ - Módulo de facturación.
        \\ \\ Tecnologías: Python, Django, CoffeeScript, Backbone.js, Bootstrap, PostgreSQL, Elasticsearch, Redis, Nginx, Linux.
      }

      ~

      \WorkEntry{\textbf{Software Developer}}
      {Petra Solar}
      {Desde Enero 2014 a Junio 2016 (2.5 años)}
      {
        Desarrollé una webapp para administrar remotamente un sistema de luces de la vía pública.
        \\ \\ Responsabilidades:
        \\ - Programación frontend.
        \\ - Reuniones diarias con el product owner (en inglés).
        \\ - Coordinación con el equipo de backend para la integración.
        \\ \\ Tecnologías: CoffeeScript, Backbone.js, D3.
      }

      ~

      \WorkEntry{\textbf{Software Developer \& Co-Fundador}}
      {Mindpulse}
      {Desde Octubre 2009 a Mayo 2011 (1.5 años)}
      {
        Co-fundé Mindpulse, mi primer empresa, durante el inicio del fin de Flash. Durante este período transicioné rápidamente a tecnologías abiertas e incorporé Python como mi lenguaje de backend. Desarrollé principalmente sitios web, juegos y aplicaciones interactivas para eventos.
        \\ \\ Tecnologías: Python, Django, JavaScript, jQuery, HTML, CSS, Flash, ActionScript, Linux.
      }

      ~

      \WorkEntry{\textbf{Software Developer}}
      {Freelance}
      {Desde Enero 2005 a Abril 2017 (12 años)}
      {
        Desarrollé más de 100 proyectos como freelancer, la mayoría sitios web, webapps, aplicaciones flash y juegos.
        Trabajé con muchas agencias de marketing / publicidad de forma recurrente realizando proyectos para grandes marcas.
        También trabajé de forma directa con una variedad de clientes más pequeños.
        \\ \\ Tecnologías: Python, Django, PHP, JavaScript, CoffeeScript, Backbone.js, HTML, CSS, Flash, ActionScript, PostgreSQL, MySQL, Redis, Nginx, Linux.
      }

    \end{itemize}

  }

\end{enumerate}
